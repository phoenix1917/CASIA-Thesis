\chapter{作者简历及攻读学位期间发表的学术论文与研究成果}

\section*{作者简历}

\noindent 黄国平,男,福建省屏南县人,1987.04.15出生,中国科学院数学与系统科学研究院博士研究生。

\noindent 2008.09-2012.06 \ \ 西南大学计算机科学与技术专业,获工学学士学位

\noindent 2012.09-2017.07 \ \ 中国科学院自动化研究所模式识别与智能系统专业,硕博连读

\section*{已发表(或正式接受)的学术论文:}

\begin{enumerate}[(1)]
	\item \textbf{Guoping Huang}, Jiajun Zhang, Yu Zhou, Chengqing Zong. A New Input Method for Human Translators: Integrating Machine Translation Effectively and Imperceptibly. In \textit{Proceedings of the 24th International Joint Conference on Artificial Intelligence (IJCAI 2015)}, Buenos Aires, Argentina, 25 July - 1 August 2015, pp. 1163-1169.

	\item \textbf{Guoping Huang}, Jiajun Zhang, Yu Zhou, Chengqing Zong. A Simple, Straightforward and Effective Model for Joint Bilingual Terms Detection and Word Alignment in SMT. In \textit{Proceedings of the Fifth Conference on Natural Language Processing and Chinese Computing \& The Twenty Fourth International Conference on Computer Processing of Oriental Languages (NLPCC 2016)}. pp. 103-115. (推荐期刊IEEE-TASLP)

	\item \textbf{Guoping Huang}, Jiajun Zhang, Yu Zhou, Chengqing Zong. Learning from User Feedback for Machine Translation in Real-Time. In \textit{Proceedings of the Fifth Conference on Natural Language Processing and Chinese Computing \& The Twenty Fourth International Conference on Computer Processing of Oriental Languages (NLPCC 2016)}, pp. 595-607.

	\item \textbf{Guoping Huang}, Chunlu Zhao, Hongyuan Ma, Jiajun Zhang, Yu Zhou. MinKSR: A Novel MT Evaluation Metric for Coordinating Human Translators with the CAT-oriented Input Method. Machine Translation. \textit{CWMT 2016. Communications in Computer and Information Science (CWMT 2016)}, vol 668, pp. 1-13. Springer, Singapore

	\item \textbf{Guoping Huang}, Jiajun Zhang, Yu Zhou, Chengqing Zong. Learning from Parenthetical Sentences for Term Translation in Machine Translation. In \textit{Proceedings of CIPS-SIGHAN Joint Conference on Chinese Language Processing 2016 (CLP 2016)}.
\end{enumerate}

\noindent 已投稿的论文:

\begin{enumerate}[(1)]
	\item \textbf{Guoping Huang}, Jiajun Zhang, Yu Zhou, Chengqing Zong. Input Method for Human Translators: A Novel Approach to Integrate Machine Translation Effectively and Imperceptibly, 已投ACM-TALLIP.
\end{enumerate}

\section*{申请或已获得的专利:}

\begin{enumerate}[(1)]
	\item 宗成庆, 黄国平, 陈宏斌, 许舟军, 王琛, 胡晨. 面向计算机辅助翻译的机器翻译译文自动评价方法与装置. 受理号: 201418010514.6 (国防专利)

	\item 宗成庆, 黄国平. 面向计算机辅助翻译的输入方法与装置. 受理号: 201410678005.X

	\item 张家俊, 黄国平, 宗成庆. 同时识别双语术语与词对齐的实现方法和实现系统. 受理号: 201611170300.X

	\item 张家俊, 黄国平, 宗成庆. 基于锚点的增长式实时双语词对齐的对齐方法和对齐系统. 受理号: 201611169586.X

	\item 张家俊, 黄国平, 宗成庆. 人机交互翻译模型的更新方法及更新系统. 受理号: 201611170954.2
\end{enumerate}

\section*{参加的研究项目及获奖情况:}

\begin{enumerate}[(1)]
	\item 国家自然科学基金重点项目:“汉语多层次语篇分析理论方法研究与应用”,项目编号:61333018

	\item 国家自然科学基金面上项目:“基于弱监督的神经网络翻译模型研究”,项目编号:61673380

	\item 国家自然科学基金“视听觉信息的认知计算”重点项目:面向汉语文本理解的语义计算方法,项目编号:91520204

	\item 国家自然科学基金:“面向篇章翻译的关键技术研究与实现”,项目编号:61403379

	\item 国家自然科学基金青年基金项目:“面向辅助翻译的多模型融合方法研究”,项目编号:61402478

	\item 中国科学院西部行动计划项目:“维汉机器翻译模型与训练解码算法”,项目编号:KGZD-EW-501

	\item 中国信息安全测评中心项目:“多语种信息采集处理与分析技术研究”
\end{enumerate}

\chapter{致\quad 谢}

值此论文完成之际,谨在此向多年来给予我关心和帮助的老师、学长、同学、
朋友和家人表示衷心的感谢!
